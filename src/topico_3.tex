% Universidade Aberta
% Template TeX para relatório de trabalhos
% 2025
%
%
% Dados para a capa
\newcommand{\Titulo}{Planeamento e Desenvolvimento de Sistemas de Informação}
\newcommand{\SubTitulo}{Trabalho Grupo - Topico 3}
\newcommand{\Ano}{2025}
\newcommand{\Autor}{
    Pedro Morais - 2401849 \\
    Hugo Gonçalves - 2100562 \\
    Pedro Moro - 2001642 \\
    Luis Peixoto - 2402741 \\
}
%
%
\documentclass[12pt,a4paper,final]{article}
\usepackage{csquotes}
\usepackage[portuguese]{babel}
\usepackage{polyglossia}
\setdefaultlanguage{portuguese}
\usepackage{graphicx}
\graphicspath{ {./images/} }
\usepackage[a4paper,top=3cm,bottom=3cm,left=3.5cm,right=2cm]{geometry}
\usepackage{booktabs}
\setmainfont{Times New Roman}
\defaultfontfeatures{Ligatures=TeX}
\usepackage[pdfauthor=\Autor,
    pdftitle=\Titulo,
    colorlinks=true,
    linkcolor=black,
    citecolor=black,
    bookmarksopen=true]{hyperref}
\hypersetup{colorlinks, citecolor=black, urlcolor=black}
\usepackage{bookmark}
\usepackage[style=apa, backend=biber, sortcites, url=true, language=portuguese]{biblatex}
\DeclareLanguageMapping{portuguese}{portuguese-apa}
\addbibresource{ref.bib}
\renewcommand{\baselinestretch}{1.5}
\begin{document}
    \title{\Titulo}
    \author{\Autor}
    \date{\Ano}
    \pagenumbering{gobble}
    \begin{titlepage}
        \begin{center}
            \vspace*{4cm}

            \textbf{\large UNIVERSIDADE ABERTA}

            \textbf{\large UNIVERSIDADE DE TRÁS-OS-MONTES E ALTO DOURO}

            \vspace{1cm}

            \begin{minipage}{0.4\textwidth}
                \centering
                \includegraphics[width=0.8\textwidth]{uab}
            \end{minipage}
            \begin{minipage}{0.4\textwidth}
                \centering
                \includegraphics[width=0.8\textwidth]{utad}
            \end{minipage}

            \vspace{1.5cm}

            \textbf{\large \Titulo}

            \textbf{\large \SubTitulo}

            \vspace{1.5cm}

            \textbf{\large \Autor}

            \vspace{2cm}

            \textbf{\large Mestrado em Engenharia Informática e Tecnologia Web}
            \vfill
            \textbf{\Ano}
        \end{center}
    \end{titlepage}
    \renewcommand{\contentsname}{Índice}
    \cleardoublepage
    \pagenumbering{roman}
    \tableofcontents
    \newpage
    \listoffigures
    \newpage
    \listoftables
    \newpage
    \cleardoublepage
    \pagenumbering{arabic}


    \section{Introdução}\label{sec:introducao}
    Este relatório descreve a arquitetura empresarial atual (“AS-IS”) da Direção Municipal de Higiene Urbana (DMHU), entidade da Câmara Municipal de Lisboa (CML), com base na documentação fornecida no âmbito da unidade curricular de Planeamento e Desenvolvimento de Sistemas de Informação (PDSI). O objetivo é estabelecer uma base sólida para o desenvolvimento futuro da arquitetura “TO-BE”, com foco na transformação digital no contexto de uma Lisboa enquanto Smart City.


    \section{Missão e Estratégia}\label{sec:missao-e-estrategia}
    A missão da DMHU é assegurar a gestão do sistema integrado de resíduos urbanos do município de Lisboa e gerir a frota municipal, promovendo a sustentabilidade e a qualidade de vida urbana.

    A estratégia da unidade integra-se na estratégia geral da CML, destacando-se os seguintes objetivos estratégicos:
    \begin{itemize}
        \item Estudos e procedimentos para uma cidade sustentável;
        \item Políticas de manutenção do espaço público;
        \item Sustentabilidade ambiental com reciclagem e reutilização;
        \item Renovação da frota com energias alternativas;
        \item Ferramentas de gestão para otimizar recursos.
    \end{itemize}


    \section{Organização}\label{sec:organizacao}
    A DMHU está organizada em duas grandes estruturas: o \textbf{Departamento de Higiene Urbana (DHU)} e o \textbf{Departamento de Reparação e Manutenção Mecânica (DRMM)}.
    Estes são apoiados por sete estruturas superiores, como o Gabinete de Apoio à Direção (GADM), Núcleo Jurídico (NJ), Núcleo de Gestão do Orçamento (NGOC), Núcleo de Armazéns (NA), entre outros.

    \subsection{Atores Internos e Externos}\label{subsec:atores-internos-e-externos}
    Os atores internos incluem colaboradores das várias unidades e departamentos da DMHU. Os atores externos envolvem:
    \begin{itemize}
        \item Cidadãos (através de plataformas como “Na Minha Rua Lx”);
        \item Juntas de Freguesia;
        \item Entidades reguladoras como a ERSAR;
        \item Prestadores de serviços e empreiteiros;
        \item Entidades municipais parceiras (como GOPI, DSI, etc.).
    \end{itemize}


    \section{Processos de Negócio}\label{sec:processos-de-negocio}
    A DMHU possui cinco processos principais:
    \begin{enumerate}
        \item \textbf{Gestão da Recolha de Resíduos} — inclui planeamento de circuitos, gestão de contentores, execução da recolha e registo de dados;
        \item \textbf{Gestão da Relação com o Cidadão} — interação com o GOPI e análise de pedidos;
        \item \textbf{Gestão de Recursos Humanos} — cadastro, ocorrências, formação, fardamento, desempenho, férias;
        \item \textbf{Gestão do Armazém} — pedidos de materiais, controlo de stocks;
        \item \textbf{Gestão da Frota} — coordenação com outras unidades municipais, manutenção e operacionalidade.
    \end{enumerate}


    \section{Informação e Aplicações}\label{sec:informacao-e-aplicacoes}
    Os principais fluxos de informação incluem:
    \begin{itemize}
        \item Ordens de trabalho e folhas de consulta;
        \item Equipamentos (viaturas, máquinas, EPI’s);
        \item Ocorrências e pedidos dos cidadãos;
        \item Registos de funcionários e sua performance;
        \item Gestão de stocks e armazém.
    \end{itemize}

    \subsection{Sistemas de Informação Existentes}\label{subsec:sistemas-de-informacao-existentes}
    \begin{itemize}
        \item \textbf{Na Minha Rua Lx} — plataforma de interação com o cidadão;
        \item \textbf{LxRequests} — gestão de pedidos e reclamações;
        \item \textbf{RH2011} — gestão administrativa de RH e formação;
        \item \textbf{Relógio Ponto} — controlo de assiduidade;
        \item \textbf{SST} — sistema de segurança e saúde no trabalho;
        \item \textbf{LU (Urban Clean)} — gestão de circuitos e equipamentos.
    \end{itemize}


    \section{Questões}\label{sec:questoes}

    \subsection{O Contexto da DMHU}\label{subsec:o-contexto-da-dmhu}
    ......

    \subsection{Produtos e Serviços de Negócio}\label{subsec:produtos-e-servicos-do-negocio}
    ......

    \subsection{Estrutura da Organização}\label{subsec:estrutura-da-organizacao}
    ......

    \subsection{Relação entre Processos e Serviços de Negócio}\label{subsec:relacao-entre-processos-e-servicos-de-negocio}
    ......

    \subsection{Visão Geral dos Processos de Negócio}\label{subsec:visao-geral-dos-processos-de-negocio}
    ......


    \section{Considerações Finais}\label{sec:consideracoes-finais}
    Esta descrição detalhada da arquitectura empresarial da DMHU fornece uma base concreta para a transição futura da organização para uma arquitetura “TO-BE”. Identificaram-se os principais fluxos de informação, processos e atores, bem como os sistemas de suporte utilizados, essenciais para delinear o plano estratégico e tecnológico para uma Lisboa mais eficiente e sustentável.

    \newpage
    % Mover referencias para ref.bib
    \section*{Referências Bibliográficas}
    \begin{itemize}
        \item Câmara Municipal de Lisboa. \textit{Relatório de Atividades 2015 da DMHU}.
        Lisboa, 2015.
        \item Direção Municipal de Higiene Urbana. \textit{Organização da DMHU 2019}.
        \item OpenGroup. \textit{Archimate 2.1 Specification}.
        Disponível em: \url{https://pubs.opengroup.org/architecture/archimate2-doc/chap08.html}
    \end{itemize}

    \newpage
    \printbibliography


\end{document}
